\chapter{Introdução}


\section{Contextualização}
Grandes transformações vêm acontecendo no mundo dos negócios e de forma tão rápida que influenciam diretamente o ambiente empresarial, levando as empresas a elaborar continuamente estratégias em busca de novas formas de se manterem competitivas no mercado, buscando avanços e melhorias tecnológicas que as coloquem à frente de seus concorrentes \cite{feres}.

Esses concorrentes estão em qualquer lugar do mundo, não bastando somente atribuir bons preços aos produtos e serviços oferecidos, é necessário algo mais. Dessa forma, investe-se alto em novas tecnologias com o intuito de tornar os processos mais eficientes e eficazes, buscando ao máximo a redução de custos, tornando-se prioridade a otimização dos investimentos e a minimização dos riscos \cite{feres}.

A partir desse cenário, algumas empresas passaram a dar mais atenção para seus ativos de Tecnologia da Informação (TI), entendendo o quanto é relevante a adoção de processos que as auxiliarão na mensuração e controle desses ativos, e é neste momento que surge a Gestão de Ativos de TI.

Aliado a esse contexto e a Governança TI, o ITIL (\textit{Information Technology Infrastructure Library} - Biblioteca de Infraestrutura de Tecnologia da Informação) apresenta uma biblioteca composta das melhores práticas para Gerenciamento de Serviços de TI. Esse conjunto de melhores práticas se tornou o \textit{framework} mais utilizado na área de TI das organizações, sendo utilizado como base para implantação do Gerenciamento de Serviços de TI em instituições públicas e privadas \cite{feres}.

Dentro da estrutura e dos processos apresentados pelo ITIL, tem-se o Processo de Ativos de Serviços e Gerenciamento de Configuração, abordado no livro de Transição de Serviço. Este processo discorre sobre o gerenciamento dos itens de configuração e sobre a importância de a organização conhecer os ativos que possui.

As ações realizadas e descritas neste relatório foram embasadas nesse processo.

\section{Objetivos}

\subsection{Objetivo Geral}
Propor ao CPD da faculdade UnB Gama melhorias no que diz respeito a gestão de ativos de TI.

\subsection{Objetivos Específicos}
\begin{itemize}
\item Efetuar listagem de ativos de TI da Faculdade UnB Gama.
\item Realizar estudo bibliográfico no que diz respeito a gestão de ativos de TI no ITIL.
\item Efetuar análise do estado atual da gestão de ativos de TI.
\end{itemize}