\chapter{Considerações Finais}
Comparando os objetivos específicos apresentados com o protótipo final, é possível perceber que o CPR atende parcialmente os objetivos. Até o fechamento do presente relatório, o robô é capaz de realizar aspiração automática de resíduos decantados e submergir de forma independente, entretanto, a movimentação está comprometida. Com as alterações realizadas nos modelos desenvolvidos, o protótipo atual, CPR-03, é capaz de submergir sozinho, realizar o acionamento automático das bombas, girar o duto de locomoção de acordo com a programação prévia dos sensores, entretanto, não é capaz de se movimentar ao fundo de piscinas somente com seu próprio sistema de bombeamento de água. Ressalta-se que a equipe espera solucionar esse problema até a apresentação, de forma a atender todos os objetivos específicos definidos inicialmente.

O grupo obteve dificuldades na etapa de integração em realizar os testes por não disponibilidade de piscinas. Como os testes foram realizados na casa de conhecidos, a equipe precisou aguardar protótipos estáveis para realização de testes. O tempo gasto na caixa de vedação e a disponibilidade limitada de acesso às piscinas dificultou a realização dos testes. A equipe mostrou-se determinada a responder as dificuldades, entretanto, cada alteração na estrutura e construção de um novo protótipo gerava tempo com secagem de nova vedação de alguns componentes e uma necessidade de agendamento para uso de uma nova piscina. Além disso, cada problema encontrado nos testes invalidava a continuação do uso do robô. Dessa forma, a equipe apenas conseguia descobrir e atacar um problema por vez. Houve aprendizado e progresso significativo em cada protótipo, entretanto, não houve alcanço do produto idealizado.

O time mostrou-se integrado na resolução de problemas. Embora houve dedicação não uniforme entre os membros, todos buscaram solucionar os problemas dos protótipos em tempo hábil para um novo teste.
A lógica do robô mostrou-se capaz de utilizar os serviços de sensoriamento, controle dos servos e ativação da bomba. Os demais serviços foram implementados e testados fora do robô, com a finalidade de garantir seu funcionamento no momento do uso. A lógica utiliza as duas centrais de processamento, Arduíno e \textit{Raspberry pi}, com a efetiva comunicação entre eles. A lógica do percurso completo utiliza máquina de estados e a camada de serviço, entretanto, o funcionamento da mesma não foi testado no robô. Os circuitos também foram construídos de forma a atender os protótipos permitindo a comunicação entre \textit{software} e o \textit{hardware}.

O protótipo atual não corresponde ao planejado, entretanto, a equipe centralizou esforços para entrega de um sistema que integrasse todas as engenharias e fosse capaz de realizar o objetivo geral do projeto: limpar piscina. Embora o último protótipo não atenda o objetivo completo, o mesmo garante a vedação do circuito, isolamentos dos componentes, submersão independente e integração mínima entre trabalhos das diversas engenharias.