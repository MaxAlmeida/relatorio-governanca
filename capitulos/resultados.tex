\chapter{Resultados}
A partir dos dados referentes aos ativos de TI, foi proposta pela equipe deste trabalho o desenvolvimento de uma solução de software voltada para gestão destes ativos, dado que até o momento todos os dados eram armazenados em planilhas. Assim, foi desenvolvida uma solução inicial com funcionalidades básicas de inserção, remoção e listagem de ativos de TI. Para o desenvolvimento dessa solução optou-se pelo uso do \textit{framework Ruby On Rails}, dado que este é conhecido por promover um rápido de desenvolvimento de soluções web e, também, pela facilidade de disponibilização das aplicações desenvolvidas sob este \textit{framework}. A planilha com os dados já listados foi convertida para o formato csv, de modo esse foi utilizado para popular a base de dados da aplicação. Assim, foi possível demonstrar que utilizando as ferramentas apresentadas, é possível desenvolver um software de maneira rápida que atenda a necessidade de gerir os ativos de TI.

Para o desenvolvimento deste trabalho, o CPD da FGA forneceu uma lista com todos os ativos campus, que incluía uma série de dados alheios aos ativos de TI e, portanto, irrelevantes para o trabalho em questão. Afim de promover o desenvolvimento de um catálogo de itens de configuração mais coerente, e com a intenção de eliminar todos os dados desnecessários, foi realizado um tratamento desses dados que consistiu em uma análise e filtragem dos mesmos.