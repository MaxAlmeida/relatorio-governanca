\chapter{Lições Apreendidas}
Ao longo do desenvolvimento do projeto nos deparamos com alguns empecilhos, como a falta de acesso aos ativos de TI relacionados à Software, devido à preocupação com a segurança, e o curto tempo para desenvolvimento de um projeto mais complexo. O impacto destes no projeto foi a falta de completude dos ativos de TI levantados e a não utilização de alguns princípios de Gerência de Configuração, como controle de mudanças relacionado aos ativos.

O controle realizado pelo CPD atualmente dos ativos de TI, mesmo que em formato mais rústico, devido a utilização de planilhas para controle dos mesmos, auxiliou o levantamento dos dados utilizados no projeto, pois as informações contidas eram precisas e organizadas.

Como objetivo de melhoria para futuros projetos na área podem ser mencionados o desenvolvimento mais extensivo de aplicação para controle dos ativos, inclusão dos ativos relacionados à Software e utilização de controle de mudanças para auxiliar o trabalho dos membros do CPD.
