\chapter{Referencial Teórico}
Para a melhor compreensão dos assuntos abordados neste trabalho e de acordo com o ITIL, é importante fazer uma distinção entre ativos de serviço e itens de configuração, que são conceitos confundidos com frequência. Um ativo de serviço é qualquer recurso ou capacidade que poderia contribuir para a prestação de um serviço, como por exemplo um servidor virtual ou físico, ou ainda um pedaço de informação na cabeça de um gerente sênior. Já um item de configuração é um ativo de serviço que precisa ser gerido com a intenção de oferecer um serviço de TI. Todos os itens de configuração são ativos de serviço, mas muitos ativos de serviço não são itens de configuração. Um exemplo de item de configuração é uma licença	 de software ou servidor.

Com base no ITIL V3 - Transição de Serviço, nenhuma organização pode ser totalmente efetiva e eficiente sem possuir uma boa gerência de seus ativos de TI, principalmente aqueles ativos que são vitais para o negócio. Toda esta gerência está bem estruturada no tópico Ativos de Serviço e Gerenciamento de Configuração (SACM).

Alguns objetivos do gerenciamento de configuração são:

\begin{itemize}
\item Dar suporte a muitos processos do ITIL provendo informações precisas sobre determinadas configurações que auxiliem na tomada de decisões. Exemplo: autorização de mudanças e planejamento de \textit{releases}.
\item Minimizar o número de problemas de qualidade e de conformidade causados ​​pela configuração incorreta ou imprecisa de serviços e bens.
\item Definir e controlar os componentes de serviços e infra-estrutura mantendo informações de configuração precisas sobre o estado anterior, planejado e atual dos mesmos.
\end{itemize}

A SACM, por sua vez, possui um propósito de assegurar que os ativos necessários para entregar o serviço são controlados de maneira correta e que as informações dos mesmos sempre sejam precisas e confiáveis quando necessário. Tais informações incluem detalhes de como os ativos estão sendo configurados e suas relações entre outros ativos.

Os ativos de serviço que precisam ser gerenciados em ordem de entregar algum serviço são chamados de itens de configuração, os quais devem ser controlados pela gerência de configuração.

Itens de configuração podem variar em complexidade, tamanho e tipo, variando de um serviço como um todo ou um sistema incluindo hardware, software, documentação e equipe de suporte para um módulo de software ou um componente menor de um hardware. Além disso, podem ser agrupados e geridos em conjunto. Estes itens devem ser selecionados com base em critérios de seleção estabelecidos, agrupados, classificados e identificados de tal maneira que são gerenciáveis e rastreáveis ao longo do ciclo de vida do serviço.

É importante para o negócio sempre estar otimizando a performance de seus ativos de serviço e suas configurações, buscando reduzir custos e riscos desnecessários. A SACM provê uma maior visibilidade das representações precisas de um serviço, permitindo assim:

\begin{itemize}
\item Que a equipe de TI entenda a configuração e relacionamento dos serviços, assim como os itens de configuração que os proveem
\item Uma melhor previsão e planejamento de mudanças
\item A resolução de incidentes e problemas dentro das metas de nível de serviço
\item A rastreabilidade de mudança de requisitos
\item A redução de custo e tempo para descobrir informações de configuração quando for necessário
\end{itemize}
