\chapter{Metodologia}
Para realização deste trabalho foram definidas reuniões semanais que poderiam acontecer em conjunto com o CPD ou apenas com os integrantes do grupo. As reuniões com o CPD tinham o propósito de obter informações inerentes ao contexto deste trabalho, ou seja, relacionadas a como o gerenciamento de ativos era realizado, se existiam ferramentas que auxiliavam nessa atividade, etc. Já as reuniões só com os integrantes do grupo foram realizadas para análise do material obtido, divisão de trabalho e tomadas de decisões.

Após a realização de uma reunião inicial com o CPD, foi obtida uma planilha contendo uma relação dos ativos de TI, que serviu de insumo para as atividades seguintes. Depois, numa reunião do grupo foram definidas três atividades principais, levando em conta o prazo para a entrega do trabalho e que o produto gerado pudesse ser útil ao CPD. As três atividades definidas foram: pesquisar sobre como é feito o gerenciamento de ativos de TI no ITIL; implementar um filtro na tabela de ativos obtida; e estruturar e escrever o relatório final das ações realizadas.

Por fim, e após a realização das duas primeiras atividades, o grupo se reuniu para executar a última atividade, compartilhar os resultados e concluir o trabalho.